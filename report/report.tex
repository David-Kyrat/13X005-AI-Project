% Options for packages loaded elsewhere
\PassOptionsToPackage{unicode}{hyperref}
\PassOptionsToPackage{hyphens}{url}
\PassOptionsToPackage{dvipsnames,svgnames,x11names}{xcolor}
%
\documentclass[
]{article}
\usepackage{amsmath,amssymb}
\usepackage{iftex}
\ifPDFTeX
  \usepackage[T1]{fontenc}
  \usepackage[utf8]{inputenc}
  \usepackage{textcomp} % provide euro and other symbols
\else % if luatex or xetex
  \usepackage{unicode-math} % this also loads fontspec
  \defaultfontfeatures{Scale=MatchLowercase}
  \defaultfontfeatures[\rmfamily]{Ligatures=TeX,Scale=1}
\fi
\usepackage{lmodern}
\ifPDFTeX\else
  % xetex/luatex font selection
\fi
% Use upquote if available, for straight quotes in verbatim environments
\IfFileExists{upquote.sty}{\usepackage{upquote}}{}
\IfFileExists{microtype.sty}{% use microtype if available
  \usepackage[]{microtype}
  \UseMicrotypeSet[protrusion]{basicmath} % disable protrusion for tt fonts
}{}
\makeatletter
\@ifundefined{KOMAClassName}{% if non-KOMA class
  \IfFileExists{parskip.sty}{%
    \usepackage{parskip}
  }{% else
    \setlength{\parindent}{0pt}
    \setlength{\parskip}{6pt plus 2pt minus 1pt}}
}{% if KOMA class
  \KOMAoptions{parskip=half}}
\makeatother
\usepackage{xcolor}
\setlength{\emergencystretch}{3em} % prevent overfull lines
\providecommand{\tightlist}{%
  \setlength{\itemsep}{0pt}\setlength{\parskip}{0pt}}
\setcounter{secnumdepth}{-\maxdimen} % remove section numbering
\usepackage{preamble_ai_project}
\usepackage[backend=bibtex,style=numeric]{biblatex}
\bibliography{references}
\ifLuaTeX
  \usepackage{selnolig}  % disable illegal ligatures
\fi
\IfFileExists{bookmark.sty}{\usepackage{bookmark}}{\usepackage{hyperref}}
\IfFileExists{xurl.sty}{\usepackage{xurl}}{} % add URL line breaks if available
\urlstyle{same}
\hypersetup{
  colorlinks=true,
  linkcolor={Blue},
  filecolor={Maroon},
  citecolor={Blue},
  urlcolor={Blue},
  pdfcreator={LaTeX via pandoc}}

\author{}
\date{}

\begin{document}

\intro{}

\hypertarget{introduction}{%
\section{1 -- Introduction}\label{introduction}}

Dans ce document, nous approfondirons des techniques de regression
logistique et ``Naive Bayes'' comme outils d'apprentissage superivisés.

Citation Test: \cite{LinearModels}

\hypertarget{muxe9thodologie}{%
\section{2 -- Méthodologie}\label{muxe9thodologie}}

\begin{itemize}
\tightlist
\item
  Outils utilisées:

  \begin{itemize}
  \tightlist
  \item
    \href{https://www.python.org/}{python}
  \item
    \href{https://scikit-learn.org/stable/}{sklearn}
  \end{itemize}
\end{itemize}

\hypertarget{impluxe9mentation}{%
\section{3 -- Implémentation}\label{impluxe9mentation}}

\hypertarget{ruxe9sultats}{%
\section{4 -- Résultats}\label{ruxe9sultats}}

\printbibliography[heading=bibintoc, title={Références}]

\begin{itemize}
\tightlist
\item
  TODO: ajouter les autres références des documentations utilisées
\end{itemize}

\end{document}
